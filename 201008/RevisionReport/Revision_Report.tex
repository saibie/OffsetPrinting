%% LyX 1.6.4 created this file.  For more info, see http://www.lyx.org/.
%% Do not edit unless you really know what you are doing.
\documentclass[10pt,a4paper,english]{amsart}
\usepackage[T1]{fontenc}
\usepackage[utf8]{inputenc}
%\usepackage{endnotes}
\usepackage{units}
%\usepackage{multirow}
\usepackage{amstext}
\usepackage{amsmath}
\usepackage{amssymb}
\usepackage{amsfonts}
\usepackage{enumerate}
\usepackage{cite}
%\usepackage{natbib}
\usepackage{amsthm}
\usepackage{array,arydshln}
\usepackage[pdftex]{graphicx}
\usepackage{rotating}
\usepackage{ifpdf}
%\usepackage{epsfig}
\usepackage[all]{xy}
\usepackage{latexsym}
\usepackage[hidelinks]{hyperref}
\usepackage{color}
\usepackage{framed}
\usepackage{ulem}
\usepackage{cancel}
\usepackage{kotex}
%\usepackage{hfont}


\makeatletter
%%%%%%%%%%%%%%%%%%%%%%%%%%%%%% Textclass specific LaTeX commands.
\numberwithin{equation}{section} %% Comment out for sequentially-numbered
\numberwithin{figure}{section} %% Comment out for sequentially-numbered
\numberwithin{table}{section}
 \let\footnote=\endnote
\theoremstyle{plain}
\newtheorem{thm}{Theorem}[section]
  \theoremstyle{definition}
  \newtheorem{defn}[thm]{Definition}
  \newtheorem{Def}[thm]{Definition}
  \newtheorem{definition}[thm]{Definition}
  \newtheorem{exam}[thm]{Example}
  \newtheorem{algo}[thm]{Algorithm}
%  \theoremstyle{plain}
  \newtheorem{assumption}[thm]{Assumption}
  \theoremstyle{plain}
  \newtheorem{lem}[thm]{Lemma}
  \newtheorem{lemma}[thm]{Lemma}
  \theoremstyle{plain}
  \newtheorem{cor}[thm]{Corollary}
  \newtheorem{corollary}[thm]{Corollary}
  \theoremstyle{plain}
  \newtheorem{rmk}[thm]{Remark}
  \newtheorem{rem}[thm]{Remark}

\def\norm#1{\|#1\|}

\newcommand\numberthis{\addtocounter{equation}{1}\tag{\theequation}}

%\newcommand{\norm}[1]{\|#1\|}
\def\norm#1{\|#1\|}
\def\normm#1#2{\|#1\|_{#2}}
\def\normF#1{\|#1\|_{F}}
\def\Proof{{\bf Proof.\enspace}}
\def\vec{\mathrm{vec}}
%\def\unvec{\mathrm{unvec}}
\def\tr{\mathrm{tr}}
%\def\tr{\textrm{tr}}
\def\bmatrix#1{\left[\begin{matrix}#1\end{matrix}\right]}
\def\pmatrix#1{\left(\begin{matrix}#1\end{matrix}\right)}
\def\R{\mathbb{R}}
\def\N{\mathbb{N}}
\def\C{\mathbb{C}}
\def\nbyn{n\times n}
\def\mbyn{m\times n}
\def\mbym{m\times m}
\def\pbyq{p\times q}
\def\nnbynn{n^{2}\times n^{2}}
\def\mbf#1{\mathbf{#1}}
\def\mrm#1{\mathrm{#1}}
\def\bpi{\boldsymbol{\pi}}
\def\a{\alpha}
\def\b{\beta}
\def\d{\delta}
\def\e{\varepsilon}
\def\l{\lambda}
\def\D{\mathcal{D}}
\def\F{\mathcal{F}}
\def\G{\mathcal{G}}
\def\Q{\mathcal{Q}}
\def\M{\mathcal{M}}
\def\P{\mathcal{P}}
\def\X{\mathcal{X}}
\def\pjn{\mathbf{P}_{\mathcal{N}}}
\def\pjm{\mathbf{P}_{\mathcal{M}}}
\def\tXi{\tilde{X}_{i}}
\def\tXii{\tilde{X}_{i+1}}
\def\dpm#1{\begin{displaymath}#1\end{displaymath}}
\def\bdm{\begin{displaymath}}
\def\edm{\end{displaymath}}
\def\beq{\begin{equation}}
\def\eeq{\end{equation}}
\def\dtyl{\displaystyle}
\def\ones#1{\mathbf{1}_{#1}}
%\def\onesn{\mathbf{1}_{n \times n}}


\makeatother



\begin{document}


\title{Revision Report}


%\begin{abstract}
%We consider the Newton iteration for a matrix polynomial equation which arises in stochastic problem.
%In this paper, we show that the elementwise minimal nonnegative solution of the matrix polynomial equation
%can be obtained using Newton's method if the equation satisfies the sufficient condition, and
%the convergence rate of the iteration is quadratic if the Fr\'{e}chet derivative at the solution is nonsingular.
%Moreover, we show that the convergence rate is at least linear if the Fr\'{e}chet derivative is singular,
%but we can apply a modified Newton method whose iteration number is less than the pure Newton iteration number.
%Finally, we give a numerical experiment which is related with our issue.
%\end{abstract}
%
%\keywords{matrix polynomial equation, elementwise positive solution, elementwise nonnegative solution, $M$-matrix, Newton's method, convergence rate, acceleration of a method}
%
%\subjclass[2010]{65H10}

\author[G. Jang]{Geun Soo Jang}
%\address{Geun Soo Jang\\
%	Finance.Fishery.Manufacture Industrial Mathematics Center on Big Data, Pusan National University, Busan, 46241, Republic of Korea}
%\email{sand621@naver.com}
\author[T. Kim]{Taehyeong Kim}
%\address{Taehyeong Kim\\
%	Finance.Fishery.Manufacture Industrial Mathematics Center on Big Data, Pusan National University, Busan, 46241, Republic of Korea}
%\email{xogud7936@pusan.ac.kr}
\author[H. Kim]{Hyun-Min Kim}
%\address{Hyun-Min Kim\\
%	Finance.Fishery.Manufacture Industrial Mathematics Center on Big Data, Pusan National University, Busan, 46241, Republic of Korea}
%\email{hyunmin@pusan.ac.kr}
\author[K. Kong]{Ki Man Kong}
%\address{Ki Man Kong\\
%	World Komax Co., Ltd., 1505, Centum Jungang-Ro 48,Haeundae-Gu, Busan, 48059, Republic of Korea}
%\email{kennethkong@worldkomax.net}
\author[J. Park]{Jeong Rye Park}
%\address{Jeong Rye Park\\
%	Finance.Fishery.Manufacture Industrial Mathematics Center on Big Data, Pusan National University, Busan, 46241, Republic of Korea}
%\email{parkjr@pusan.ac.kr}
\author[J. Seo]{Jong-hyeon Seo}
%\address{Jong-hyeon Seo\\
%	Chubu University Academy of Emerging Science, Kasugai, 487-0027, Japan}
%\email{hyeonni94@gmail.com}
\author[S. Seo]{Sang-hyup Seo$^{\dagger}$}
%\address{Sang-hyup Seo\\
%	Finance.Fishery.Manufacture Industrial Mathematics Center on Big Data, Pusan National University, Busan, 46241, Republic of Korea}
%\email{saibie1677@gmail.com}
\author[S. Yoon]{Shin won Yoon}
%\address{Shin won Yoon\\
%	Finance.Fishery.Manufacture Industrial Mathematics Center on Big Data, Pusan National University, Busan, 46241, Republic of Korea}
%\email{ysw0123@pusan.ac.kr}


%\thanks{$\dagger$Corresponding Author}
%
%\thanks{This research was supported by Basic Science Research Program through
%		the National Research Foundation of Korea(NRF) funded by the Ministry
%		of Education, Science and Technology(2017R1A5A1015722, 2017R1D1A3B04033516)}

\maketitle

\begin{center}
	{\LARGE \bf <Revisions>}
\end{center}


\noindent{\bf L3--4 in Abstract :} 

\noindent It is by setting an acceptable upper limit for each plate, and carrying out

\noindent $\longrightarrow$ It is {\color{red}done} by setting an acceptable upper limit for each plate, and {\color{red}by} carrying out\\

\noindent{\bf L13 in P1:} 

\noindent two \sout{each of} A's and B's $\longrightarrow$ two A's and {\color{red}two} B's\\

\noindent{\bf L19, L21 in P1:} 

\noindent a ‘\xcancel{B}etter’ $\longrightarrow$ a ‘better’ {\color{red}choice}\\

\noindent{\bf L19--20 in P1 :} 

\noindent Let us examine \sout{the other} case. $\longrightarrow$ Let us examine {\color{red}another} case. \\

\noindent{\bf L22 in P1 :} 

\noindent $\textrm{[A, B, B, B]} \times 3$ is the ‘\sout{B}est’ \sout{method}. $\longrightarrow$ $\textrm{[A, B, B, B]} \times 3$ is the {\color{red}‘best’ choice.} \\

\noindent{\bf L23 in P1 :} 

\noindent Minimize the number of list.
$\longrightarrow$ Minimize the number of list{\color{red}s}. \\

\noindent{\bf L36--37 in P2 :} 

\noindent The above example \sout{denotes} what is the best arrangement in such print method.

\noindent $\longrightarrow$ The above example {\color{red}shows} what is the best arrangement in such print{\color{red}ing} method. \\

\noindent{\bf L45--46 in P2 :} 

\noindent and variable information such as the manufactur\sout{ed} date.

\noindent $\longrightarrow$ and variable information such as the manufactur{\color{red}ing} date. \\

\noindent{\bf L48 in P2 :} 

\noindent the number of label loss increased compared to the past.

\noindent $\longrightarrow$ the number of label loss{\color{red}es} increased compared to the past.\\

\noindent{\bf L81 in P3 :} 

\noindent input\sout{ted}
$\longrightarrow$ input \\

\noindent{\bf L92 in P3 :} 

\noindent $b_{i}$ be the number of order.
$\longrightarrow$ $b_{i}$ be the number of order{\color{red}s}. \\

\noindent{\bf L96 in P3 :} 

\noindent satisfy the following
$\longrightarrow$ satisfy{\color{red}ing} the following \\

\noindent{\bf L98 in P3 : Period is inserted.} \\

\noindent{\bf L119 in P4 : } 

\noindent input\sout{ted}
$\longrightarrow$ input \\ \vspace{2cm}

\noindent{\bf L127 in P4 : } 

\noindent $N$ is a printing number that expressed in

\noindent $\longrightarrow$ $N$ is a printing number that {\color{red}is} expressed in\\

\noindent{\bf L142 in P6 : } 

\noindent We can see \sout{the} Figure $\longrightarrow$ We can see Figure \\

\noindent{\bf (4.6) below L154 in P7 : Period is deleted.} \\

\noindent{\bf L156 in P7 : } 

\noindent because its number of plate $\longrightarrow$ because its number of plate{\color{red}s} \\

\noindent{\bf L159 in P7 : } 

\noindent Now, we consider real problem of World Komax.

$\longrightarrow$ Now, we consider {\color{red}a} real problem of World Komax. \\

\noindent{\bf The line above matrix $A$ which is above L166 in P8 : } 

\noindent The matrix $A$ can be found as follow\xcancel{ed}.

\noindent $\longrightarrow$ The matrix $A$ can be found as follow{\color{red}s}. \\

\noindent{\bf L175 in P9 : } 

\noindent The two populations are as \sout{such}. $\longrightarrow$ The two populations are as {\color{red}follows}. \\

\noindent{\bf L205 in P10 : } 

\noindent the size of $I$(cf. Section 3) 
$\longrightarrow$ the size of $I${\color{red}$\check{\phantom{a}}$}(cf. Section 3) \\

\noindent{\bf L206 in P10 : } 

\noindent For each 82 samples,

\noindent $\longrightarrow$ For each {\color{red}of the} 82 samples,

%\bibliographystyle{plain}
%\bibliography{SHSeo}

\end{document}

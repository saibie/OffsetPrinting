%% LyX 1.6.4 created this file.  For more info, see http://www.lyx.org/.
%% Do not edit unless you really know what you are doing.
\documentclass[10pt,a4paper,english]{amsart}
\usepackage[T1]{fontenc}
\usepackage[utf8]{inputenc}
%\usepackage{endnotes}
\usepackage{units}
%\usepackage{multirow}
\usepackage{amstext}
\usepackage{amsmath}
\usepackage{amssymb}
\usepackage{amsfonts}
\usepackage{enumerate}
\usepackage{cite}
%\usepackage{natbib}
\usepackage{amsthm}
\usepackage{array,arydshln}
\usepackage[pdftex]{graphicx}
\usepackage{rotating}
\usepackage{ifpdf}
%\usepackage{epsfig}
\usepackage[all]{xy}
\usepackage{latexsym}
\usepackage[hidelinks]{hyperref}
\usepackage{color}
\usepackage{framed}
\usepackage{ulem}
\usepackage{cancel}
\usepackage{kotex}
%\usepackage{hfont}


\makeatletter
%%%%%%%%%%%%%%%%%%%%%%%%%%%%%% Textclass specific LaTeX commands.
\numberwithin{equation}{section} %% Comment out for sequentially-numbered
\numberwithin{figure}{section} %% Comment out for sequentially-numbered
\numberwithin{table}{section}
 \let\footnote=\endnote
\theoremstyle{plain}
\newtheorem{thm}{Theorem}[section]
  \theoremstyle{definition}
  \newtheorem{defn}[thm]{Definition}
  \newtheorem{Def}[thm]{Definition}
  \newtheorem{definition}[thm]{Definition}
  \newtheorem{exam}[thm]{Example}
  \newtheorem{algo}[thm]{Algorithm}
%  \theoremstyle{plain}
  \newtheorem{assumption}[thm]{Assumption}
  \theoremstyle{plain}
  \newtheorem{lem}[thm]{Lemma}
  \newtheorem{lemma}[thm]{Lemma}
  \theoremstyle{plain}
  \newtheorem{cor}[thm]{Corollary}
  \newtheorem{corollary}[thm]{Corollary}
  \theoremstyle{plain}
  \newtheorem{rmk}[thm]{Remark}
  \newtheorem{rem}[thm]{Remark}

\def\norm#1{\|#1\|}

\newcommand\numberthis{\addtocounter{equation}{1}\tag{\theequation}}

%\newcommand{\norm}[1]{\|#1\|}
\def\norm#1{\|#1\|}
\def\normm#1#2{\|#1\|_{#2}}
\def\normF#1{\|#1\|_{F}}
\def\Proof{{\bf Proof.\enspace}}
\def\vec{\mathrm{vec}}
%\def\unvec{\mathrm{unvec}}
\def\tr{\mathrm{tr}}
%\def\tr{\textrm{tr}}
\def\bmatrix#1{\left[\begin{matrix}#1\end{matrix}\right]}
\def\pmatrix#1{\left(\begin{matrix}#1\end{matrix}\right)}
\def\R{\mathbb{R}}
\def\N{\mathbb{N}}
\def\C{\mathbb{C}}
\def\nbyn{n\times n}
\def\mbyn{m\times n}
\def\mbym{m\times m}
\def\pbyq{p\times q}
\def\nnbynn{n^{2}\times n^{2}}
\def\mbf#1{\mathbf{#1}}
\def\mrm#1{\mathrm{#1}}
\def\bpi{\boldsymbol{\pi}}
\def\a{\alpha}
\def\b{\beta}
\def\d{\delta}
\def\e{\varepsilon}
\def\l{\lambda}
\def\D{\mathcal{D}}
\def\F{\mathcal{F}}
\def\G{\mathcal{G}}
\def\Q{\mathcal{Q}}
\def\M{\mathcal{M}}
\def\P{\mathcal{P}}
\def\X{\mathcal{X}}
\def\pjn{\mathbf{P}_{\mathcal{N}}}
\def\pjm{\mathbf{P}_{\mathcal{M}}}
\def\tXi{\tilde{X}_{i}}
\def\tXii{\tilde{X}_{i+1}}
\def\dpm#1{\begin{displaymath}#1\end{displaymath}}
\def\bdm{\begin{displaymath}}
\def\edm{\end{displaymath}}
\def\beq{\begin{equation}}
\def\eeq{\end{equation}}
\def\dtyl{\displaystyle}
\def\ones#1{\mathbf{1}_{#1}}
%\def\onesn{\mathbf{1}_{n \times n}}


\makeatother



\begin{document}


\title{Revision Report}


%\begin{abstract}
%We consider the Newton iteration for a matrix polynomial equation which arises in stochastic problem.
%In this paper, we show that the elementwise minimal nonnegative solution of the matrix polynomial equation
%can be obtained using Newton's method if the equation satisfies the sufficient condition, and
%the convergence rate of the iteration is quadratic if the Fr\'{e}chet derivative at the solution is nonsingular.
%Moreover, we show that the convergence rate is at least linear if the Fr\'{e}chet derivative is singular,
%but we can apply a modified Newton method whose iteration number is less than the pure Newton iteration number.
%Finally, we give a numerical experiment which is related with our issue.
%\end{abstract}
%
%\keywords{matrix polynomial equation, elementwise positive solution, elementwise nonnegative solution, $M$-matrix, Newton's method, convergence rate, acceleration of a method}
%
%\subjclass[2010]{65H10}

\author[G. Jang]{Geun Soo Jang}
%\address{Geun Soo Jang\\
%	Finance.Fishery.Manufacture Industrial Mathematics Center on Big Data, Pusan National University, Busan, 46241, Republic of Korea}
%\email{sand621@naver.com}
\author[T. Kim]{Taehyeong Kim}
%\address{Taehyeong Kim\\
%	Finance.Fishery.Manufacture Industrial Mathematics Center on Big Data, Pusan National University, Busan, 46241, Republic of Korea}
%\email{xogud7936@pusan.ac.kr}
\author[H. Kim]{Hyun-Min Kim}
%\address{Hyun-Min Kim\\
%	Finance.Fishery.Manufacture Industrial Mathematics Center on Big Data, Pusan National University, Busan, 46241, Republic of Korea}
%\email{hyunmin@pusan.ac.kr}
\author[K. Kong]{Ki Man Kong}
%\address{Ki Man Kong\\
%	World Komax Co., Ltd., 1505, Centum Jungang-Ro 48,Haeundae-Gu, Busan, 48059, Republic of Korea}
%\email{kennethkong@worldkomax.net}
\author[J. Park]{Jeong Rye Park}
%\address{Jeong Rye Park\\
%	Finance.Fishery.Manufacture Industrial Mathematics Center on Big Data, Pusan National University, Busan, 46241, Republic of Korea}
%\email{parkjr@pusan.ac.kr}
\author[J. Seo]{Jong-hyeon Seo}
%\address{Jong-hyeon Seo\\
%	Chubu University Academy of Emerging Science, Kasugai, 487-0027, Japan}
%\email{hyeonni94@gmail.com}
\author[S. Seo]{Sang-hyup Seo$^{\dagger}$}
%\address{Sang-hyup Seo\\
%	Finance.Fishery.Manufacture Industrial Mathematics Center on Big Data, Pusan National University, Busan, 46241, Republic of Korea}
%\email{saibie1677@gmail.com}
\author[S. Yoon]{Shin won Yoon}
%\address{Shin won Yoon\\
%	Finance.Fishery.Manufacture Industrial Mathematics Center on Big Data, Pusan National University, Busan, 46241, Republic of Korea}
%\email{ysw0123@pusan.ac.kr}


%\thanks{$\dagger$Corresponding Author}
%
%\thanks{This research was supported by Basic Science Research Program through
%		the National Research Foundation of Korea(NRF) funded by the Ministry
%		of Education, Science and Technology(2017R1A5A1015722, 2017R1D1A3B04033516)}

\maketitle

\begin{center}
	{\LARGE \bf <Revisions>}
\end{center}


\noindent{\bf L3 in Abstract :} 

\noindent changing the array of \xcancel{the} pieces on the printing plates in the offset printing.

\noindent $\longrightarrow$ changing the array of pieces on the printing plates in the offset printing.\\

\noindent{\bf L3--4 in Abstract :} 

\noindent It is setting an upper limit \sout{of acceptable} for each plate, and carrying out complete enumeration. 

\noindent $\longrightarrow$ It is {\color{red}by} setting an {\color{red}acceptable} upper limit for each plate, and carrying out complete enumeration.\\

\noindent{\bf L5 in Abstract :} 

\noindent This method \sout{dramatically} reduces the operating time of the algorithm. 

\noindent $\longrightarrow$ This method {\color{red}drastically} reduces the operating time of the algorithm.\\

\noindent{\bf L7 in P1 :} 

\noindent selected from \sout{among} different $n$ elements

\noindent $\longrightarrow$ selected from different $n$ elements \\

\noindent{\bf L16 in P1 :} 

\noindent if we want to obtain three A's and nine B's, \xcancel{then} we can choose

\noindent $\longrightarrow$ if we want to obtain three A's and nine B's, we can choose \\

\noindent{\bf L21 in P1 :} 

\noindent $\textrm{[A, B, B, B]} \times 3$ also seems to be `Better'

\noindent $\longrightarrow$ $\textrm{[A, B, B, B]} \times 3$ also seems to be {\color{red}a} `Better' \\

\noindent{\bf L25--26 in P1 :} 

\noindent Offset printing, also called offset lithography, or litho-offset\xcancel{,} in commercial printing, widely used printing technique

\noindent $\longrightarrow$ Offset printing, also called offset lithography, or litho-offset in commercial printing, {\color{red}is a} widely used printing technique \\

\noindent{\bf L35 in P2 :} 

\noindent As \sout{an other improving}, how to make the initial plates

\noindent $\longrightarrow$ As another improvement, how to make the initial plates \\

\noindent{\bf L36 in P2 :} 

\noindent Above example \sout{means that} what is the best arrangement 

\noindent $\longrightarrow$ {\color{red}The a}bove example {\color{red}denotes} what is the best arrangement \vspace{1.5cm}

\noindent{\bf L38 in P2 :} 

\noindent In the past, production was based on ordering of products from companies 

\noindent $\longrightarrow$ In the past, production was based on {\color{red}the} ordering of products from companies \\

\noindent{\bf L39--40 in P2 :} 

\noindent However, as the internet market has \sout{became} popular, the production systems have been \sout{changed} by consumers. 

\noindent $\longrightarrow$ However, as the internet market has {\color{red}become} popular, the production systems have been {\color{red}altered} by consumers. \\

\noindent{\bf L45 in P2 :} 

\noindent variable information such as the \sout{date of manufacture}.

\noindent $\longrightarrow$ variable information such as the {\color{red}manufactured date}. \\

\noindent{\bf L47 in P2 :} 

\noindent the number of label\xcancel{s} loss increased compared to the past.

\noindent $\longrightarrow$ the number of label loss increased compared to the past. \\

\noindent{\bf L57 in P3 : } 

\noindent \sout{At first}, we receive orders $\longrightarrow$ {\color{red}First}, we receive orders \\

\noindent{\bf L60 in P3 : } 

\noindent placed on each plate so that many labels are printed at \sout{one printing}. 

\noindent $\longrightarrow$ placed on each plate so that many labels are printed at {\color{red}once}.  \\

\noindent{\bf L62 in P3 : } 

\noindent \sout{As} the final process, $\longrightarrow$ {\color{red}For} the final process, \\

\noindent{\bf L64--65 in P3 : } 

\noindent The constraint \sout{conditions} and major points 

\noindent $\longrightarrow$ The constraint{\color{red}s} and major points \\

\noindent{\bf L74 in P3 : } 

\noindent minimized as \sout{little as} $\longrightarrow$ {\color{red}as} minimized as \\

\noindent{\bf L84 in P3 : } 

\noindent the algorithm \sout{should} be improved. $\longrightarrow$ the algorithm {\color{red}had to} be improved. \\

\noindent{\bf L101 in P4 : } 

\noindent goal\xcancel{s} is to obtain the following $\longrightarrow$ goal is to obtain the following \\

\noindent{\bf L105 in P4 : } 

\noindent However, this method \sout{has a problem that it} takes too much time. 

\noindent $\longrightarrow$ However, this method takes too much time. \\

\noindent{\bf L107 in P4 : } 

\noindent Then, the calculation \sout{of the cases} takes more than 658 hours, 

\noindent $\longrightarrow$ Then, the calculation takes more than 658 hours, \vspace{2cm}

\noindent{\bf L108--109 in P4 : } 

\noindent Given that there are \sout{limits of the time} from the date of \sout{receipt of orders} to the delivery date, 

\noindent $\longrightarrow$ Given that there are {\color{red}time constraints} from the date of {\color{red}order receipts} to the delivery date, \\

\noindent{\bf L119 in P4 : } 

\noindent $P$ is the products that contain the Plate

\noindent $\longrightarrow$ $P$ is the {\color{red}set of} products that contain{\color{red}s} the Plate \\

\noindent{\bf L124 in P4 : } 

\noindent with repetition $_{num}H_{k}$ \sout{and} indicated in

\noindent $\longrightarrow$ with repetition $_{num}H_{k}$ {\color{red}which is} indicated in \\

\noindent{\bf L152 in P7 : } 

\noindent For the same partition $\pi$, \sout{the} matrix $A = (\begin{array}{cc}3 & 1\end{array})$ can be 

\noindent $\longrightarrow$ For the same partition $\pi$, matrix $A = (\begin{array}{cc}3 & 1\end{array})$ can be \\

\noindent{\bf L157 in P7 : } 

\noindent If there are \sout{so} many products and \sout{also} large order-quantit\xcancel{y}, 

\noindent $\longrightarrow$ If there are {\color{red}too} many products and large order-quantit{\color{red}ies}, \\

\noindent{\bf The line above matrix $A$ which is above L165 in P8 : } 

\noindent The matrix $A$ can be found as follow\xcancel{s}.

\noindent $\longrightarrow$ The matrix $A$ can be found as follow{\color{red}ed}. \\

\noindent{\bf L170--171 in P9 : } 

\noindent The total cost was reduced by \sout{from} minimum -6.85\%(sample no. 15) to maximum 27.5\%(sample no. 74)

\noindent $\longrightarrow$ The total cost was reduced by {\color{red}a} minimum {\color{red}of} -6.85\%(sample no. 15) to {\color{red}a} maximum {\color{red}of} 27.5\%(sample no. 74) \\

\noindent{\bf L172--174 in P9 : } 

\noindent The paired t-test is one of the two sample t-test, and it is a test that verifies whether the two groups are different. The two populations are as \sout{follows}.

\noindent $\longrightarrow$ The paired t-test is one of the two sample t-test{\color{red}s}, and it is a test that verifies whether the two groups are different. The two populations are as {\color{red}such}. \\

\noindent{\bf L179--180 in P9 : } 

\noindent the two groups have to satisfy \sout{the} normality and homoscedasticity. 

\noindent $\longrightarrow$ the two groups have to satisfy normality and homoscedasticity. \\

\noindent{\bf L181 in P9 : } 

\noindent which can \sout{be} satisf\sout{ied the} normality 

\noindent $\longrightarrow$ which can satisf{\color{red}y} normality \\

\noindent{\bf L205--206 in P10 : } 

\noindent the efficiency as the follow formula. $\longrightarrow$ the efficiency as the follow{\color{red}ing} formula. \vspace{2cm}

\noindent{\bf L209 in P10 : } 

\noindent in many \sout{number of} products. \noindent $\longrightarrow$ in many products. \\

\noindent{\bf L212 in P11 : } 

\noindent for \sout{confidentiality of the company}. \noindent $\longrightarrow$ for {\color{red}company confidentiality reasons}. \\

\noindent{\bf L217--218 in P11 :} 

\noindent who provided \sout{many} support and \sout{advises} for \sout{the} writing of this paper.

\noindent $\longrightarrow$ who provided {\color{red}much} support and {\color{red}advice} for writing of this paper. \\

%\bibliographystyle{plain}
%\bibliography{SHSeo}

\end{document}

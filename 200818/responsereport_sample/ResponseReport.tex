\documentclass[10pt]{amsart}
\usepackage{amsmath}
\usepackage{graphicx}
\usepackage{amssymb}
\usepackage{epstopdf}
\usepackage{color}
\usepackage{float}
\restylefloat{table}
\usepackage{tikz}
\usetikzlibrary{arrows}
\usetikzlibrary{matrix}
\DeclareGraphicsRule{.tif}{png}{.png}{`convert #1 `dirname #1`/`basename #1 .tif`.png}

\setlength{\parskip}{2mm}

\newtheorem{thm}{Theorem}[section]
\newtheorem{lem}[thm]{Lemma}
\newtheorem{prop}[thm]{Proposition}
\newtheorem{defn}[thm]{Definition}
\newtheorem{cor}[thm]{Corollary}
\newtheorem{conjecture}[thm]{Conjecture}
\newtheorem{exam}[thm]{Example}
\newtheorem{remark}[thm]{Remark}


\newcommand{\FN}{\mathbb{N}}
\newcommand{\FZ}{\mathbb{Z}}  % Integer ring Z
\newcommand{\FC}{\mathbb{C}}  % Complex field
\newcommand{\FR}{\mathbb{R}}  % Real field
\newcommand{\FQ}{\mathbb{Q}}  % Rational field

\newcommand{\z}{\zeta}
\newcommand{\fa}{\mathfrak{a}}
\newcommand{\fb}{\mathfrak{b}}
\newcommand{\fc}{\mathfrak{c}}
\newcommand{\fm}{\mathfrak{m}}
\newcommand{\fo}{\mathfrak{o}}
\newcommand{\fp}{\mathfrak{p}}
\newcommand{\fq}{\mathfrak{q}}

\newcommand{\h}{\mathcal{H}}
\newcommand{\F}{\mathbb{F}}
\newcommand{\cc}{\mathcal{C}}
\newcommand{\R}{\mathcal{R}}
\renewcommand{\O}{\mathcal{O}}

\newcommand{\C}{\mathbf{C}}
\newcommand{\s}{\mathbf{s}}
\newcommand{\mm}{\mathrm{m}}
\newcommand{\bx}{\mathbf{x}}
\newcommand{\bu}{\mathbf{u}}
\newcommand{\bc}{\mathbf{c}}
\newcommand{\bv}{\mathbf{v}}
\newcommand{\lc}{\Big{(}}
\newcommand{\rc}{\Big{)}}
\newcommand{\Tr}{\textnormal{Tr}}
\newcommand{\Gal}{\textnormal{Gal}}


\usepackage{kotex}
\begin{document}
\title[Response Report]
{Response Report}
\maketitle
\centerline{\bf ``Development of an Algorithm Improving Label Arrangements in Offset Printing''}
\centerline{\bf by GEUN SOO JANG, TAEHYEONG KIM, HYUN-MIN KIM, KI MAN KONG,}
\centerline{\bf  JEONG RYE PARK,JONG-HYEON SEO,SANG-HYUP SEO, AND SHIN WON YOON}

\

We thank referees for their careful review and helpful comments, which improved clarity of our paper.
\vspace{.5cm}
\begin{center}
	{\Large\bf Major Revision}
\end{center}
\begin{enumerate}

\item {\bf Suggestion for Abstract} - ``The advantage of this method is that there is only change the arrangement of the pieces on the plates.''
$~\longrightarrow~$ ``The advantage of this method is that it focuses on changing the arrangment of pieces on one plate at a time.'' :\\
~\\
We almost revised as the referee's suggestion but, in this paper, we do not talk about changing the arrangement of pieces on {\bf one plate at a time}.
So, we revised the sentence to ``The advantage of this method is that it focuses on changing the arrangement of the pieces on the plates.''\\

\item {\bf The spelling of the Figure 2.1} : \\
~\\
We revised it.\\

\item {\bf ``The algorithm can be improved for every sample data'' in Section 5} : \\
~\\
We deleted it as referee's suggestion.\\

\item {\bf Line 2 below (4.4)} : \\
~\\
Revision : ``The array shown in Figure 4.2. is a more efficient because...'' $~\longrightarrow~$ ``The array shown in Figure 4.2. is more efficient since...''

\end{enumerate}
\newpage
\vspace{.5cm}
\begin{center}
	{\Large\bf Minor Revision}
\end{center}
\begin{enumerate}

\item {\bf Line 1 of Subsection 2.1} :\\
~\\
``Prior to describing the label printing process,''\\$~\longrightarrow~$``Before discussing the process of label printing,''\\

\item {\bf Line 2 of Subsection 2.1} :\\
~\\
``terms first.''$~\longrightarrow~$``terms.''\\

\item {\bf Line 2 above Subsection 2.2} :\\
~\\
``The sheets are then cut to the sizes of labels.''$~\longrightarrow~$``Then, the sheets are cut into the label size.''\\

\item {\bf Line 1 of Subsection 2.3} :\\
~\\
``Plate fabrication''$~\longrightarrow~$``The production of plates''\\

\end{enumerate}
\end{document} 
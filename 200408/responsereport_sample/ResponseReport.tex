\documentclass[10pt]{amsart}
\usepackage{amsmath}
\usepackage{graphicx}
\usepackage{amssymb}
\usepackage{epstopdf}
\usepackage{color}
\usepackage{float}
\restylefloat{table}
\usepackage{tikz}
\usetikzlibrary{arrows}
\usetikzlibrary{matrix}
\DeclareGraphicsRule{.tif}{png}{.png}{`convert #1 `dirname #1`/`basename #1 .tif`.png}

\setlength{\parskip}{2mm}

\newtheorem{thm}{Theorem}[section]
\newtheorem{lem}[thm]{Lemma}
\newtheorem{prop}[thm]{Proposition}
\newtheorem{defn}[thm]{Definition}
\newtheorem{cor}[thm]{Corollary}
\newtheorem{conjecture}[thm]{Conjecture}
\newtheorem{exam}[thm]{Example}
\newtheorem{remark}[thm]{Remark}


\newcommand{\FN}{\mathbb{N}}
\newcommand{\FZ}{\mathbb{Z}}  % Integer ring Z
\newcommand{\FC}{\mathbb{C}}  % Complex field
\newcommand{\FR}{\mathbb{R}}  % Real field
\newcommand{\FQ}{\mathbb{Q}}  % Rational field

\newcommand{\z}{\zeta}
\newcommand{\fa}{\mathfrak{a}}
\newcommand{\fb}{\mathfrak{b}}
\newcommand{\fc}{\mathfrak{c}}
\newcommand{\fm}{\mathfrak{m}}
\newcommand{\fo}{\mathfrak{o}}
\newcommand{\fp}{\mathfrak{p}}
\newcommand{\fq}{\mathfrak{q}}

\newcommand{\h}{\mathcal{H}}
\newcommand{\F}{\mathbb{F}}
\newcommand{\cc}{\mathcal{C}}
\newcommand{\R}{\mathcal{R}}
\renewcommand{\O}{\mathcal{O}}

\newcommand{\C}{\mathbf{C}}
\newcommand{\s}{\mathbf{s}}
\newcommand{\mm}{\mathrm{m}}
\newcommand{\bx}{\mathbf{x}}
\newcommand{\bu}{\mathbf{u}}
\newcommand{\bc}{\mathbf{c}}
\newcommand{\bv}{\mathbf{v}}
\newcommand{\lc}{\Big{(}}
\newcommand{\rc}{\Big{)}}
\newcommand{\Tr}{\textnormal{Tr}}
\newcommand{\Gal}{\textnormal{Gal}}


\usepackage{kotex}
\begin{document}
\title[Response Report]
{Response Report}
\maketitle
\centerline{\bf "Development of an Algorithm Improving Label Arrangements in Offset Printing"}
\centerline{\bf by GEUN SOO JANG, TAEHYEONG KIM, HYUN-MIN KIM, KI MAN KONG,}
\centerline{\bf  JEONG RYE PARK,JONG-HYEON SEO,SANG-HYUP SEO, AND SHIN WON YOON}

\

We thank referees for their careful review and helpful comments, which improved clarity of our paper.

\

\begin{enumerate}

\item We fixed some small typos.\\

\item Page 5, after FIGURE 3.1:\\
Add more detailed description of Figure 3.1\\

\item Page 5, after "We set the result as {\it Part\underline{ }list}."\\
For example, let $k=6$ and $num=3$, then $\text{\it Part\underline{ }list}={Part}(6, 3) = \{ [4,1,1],[3,2,1],[2,2,2] \}$.
$N$ is a printing number that expressed in (3.1).\\

\item Page 5, above Example 4.1 : \\
In this section, we assume that $k$ is equal to $4$\\
$\Longrightarrow$\\
For the next two examples, we assume that $k$ is equal to $4$\\

\item page 5, In Example 4.1 : \\
$I=\{A,B,C\}$,  $\pi = \{\{A,B\}, \{C\}\}$, $P_{1} = \{A,B\}, P_{2} = \{C\}$\\
$\Longrightarrow$\\
$I=\{1,2,3\}$,  $\pi = \{\{1,2\}, \{3\}\}$, $P_{1} = \{1,2\}, P_{2} = \{3\}$\\

\item fixed FIGURE 4.1 and FIGURE 4.2 \\


\item page 6, In Example 4.2 : \\
$I=\{A,B\}$,  $\pi = \{\{A,B\}\}$ \\
$\Longrightarrow$\\
$I=\{1,2\}$,  $\pi = \{\{1,2\}\}$ \\

\item fixed FIGURE 4.3 and FIGURE 4.4 \\

\item page 6, after FIGURE 4.4 :\\
Add the real problem and Example 4.3\\

\end{enumerate}


\end{document} 
\documentclass[a4paper]{amsart}
\usepackage{units}
\usepackage{multirow}
\usepackage{amstext}
\usepackage{amsmath}
\usepackage{amssymb}
\usepackage{amsfonts}
\usepackage{enumerate}
\usepackage{cite}
%\usepackage{natbib}
\usepackage{amsthm}
\usepackage{array,arydshln}
\usepackage[pdftex]{graphicx}
\usepackage{rotating}
\usepackage{ifpdf}
%\usepackage{epsfig}
\usepackage[all]{xy}
\usepackage{latexsym}
\usepackage[hidelinks]{hyperref}
\usepackage{color}
\usepackage{wrapfig}
\usepackage{caption}
\usepackage{blkarray}
\usepackage{array}


\makeatletter
%%%%%%%%%%%%%%%%%%%%%%%%%%%%%% Textclass specific LaTeX commands.
\numberwithin{equation}{section} %% Comment out for sequentially-numbered
\numberwithin{figure}{section} %% Comment out for sequentially-numbered
\numberwithin{table}{section}
\let\footnote=\endnote
\theoremstyle{plain}
\newtheorem{thm}{Theorem}[section]
\theoremstyle{definition}
\newtheorem{defn}[thm]{Definition}
\newtheorem{Def}[thm]{Definition}
\newtheorem{definition}[thm]{Definition}
\newtheorem{exam}[thm]{Example}
\newtheorem{example}[thm]{Example}
\newtheorem{algo}[thm]{Algorithm}
\newtheorem{note}[thm]{Note}
\theoremstyle{plain}
\newtheorem{assumption}[thm]{Assumption}
\theoremstyle{plain}
\newtheorem{lem}[thm]{Lemma}
\newtheorem{lemma}[thm]{Lemma}
\theoremstyle{plain}
\newtheorem{cor}[thm]{Corollary}
\newtheorem{corollary}[thm]{Corollary}
\theoremstyle{plain}
\newtheorem{rmk}[thm]{Remark}
\newtheorem{rem}[thm]{Remark}
\theoremstyle{plain}
\newtheorem{Proposition}[thm]{Proposition}
\newtheorem{pro}[thm]{Proposition}


%\newcommand{\norm}[1]{\|#1\|}
\def\norm#1{\|#1\|}
\def\normm#1#2{\|#1\|_{#2}}
\def\normF#1{\|#1\|_{F}}
\def\Proof{{\bf Proof.\enspace}}
\def\vec{\mathrm{vec}}
%\def\unvec{\mathrm{unvec}}
\def\tr{\mathrm{tr}}
%\def\tr{\textrm{tr}}
\def\bmatrix#1{\left[\begin{matrix}#1\end{matrix}\right]}
\def\pmatrix#1{\left(\begin{matrix}#1\end{matrix}\right)}
\def\R{\mathbb{R}}
\def\N{\mathbb{N}}
\def\C{\mathbb{C}}
\def\nbyn{n\times n}
\def\mbyn{m\times n}
\def\mbym{m\times m}
\def\pbyq{p\times q}
\def\nnbynn{n^{2}\times n^{2}}
\def\mbf#1{\mathbf{#1}}
\def\mrm#1{\mathrm{#1}}
\def\bpi{\boldsymbol{\pi}}
\def\a{\alpha}
\def\b{\beta}
\def\d{\delta}
\def\e{\varepsilon}
\def\l{\lambda}
\def\D{\mathcal{D}}
\def\F{\mathcal{F}}
\def\G{\mathcal{G}}
\def\Q{\mathcal{Q}}
\def\M{\mathcal{M}}
\def\P{\mathcal{P}}
\def\dpm#1{\begin{displaymath}#1\end{displaymath}}
\def\bdm{\begin{displaymath}}
\def\edm{\end{displaymath}}
\def\beq{\begin{equation}}
\def\eeq{\end{equation}}
\def\dtyl{\displaystyle}
\def\ones#1{\mathbf{1}_{#1}}
%\def\onesn{\mathbf{1}_{n \times n}}

\makeatother

\title[Development of an Algorithm Improving Label Arrangements]{Development of an Algorithm Improving Label Arrangements in Offset Printing}

%\authorrunning{Short form of author list} % if too long for running head

\author[G. S. Jang]{Geun Soo Jang}
\address{Geun Soo Jang\\
	Finance.Fishery.Manufacture Industrial Mathematics Center on Big Data, Pusan National University, Busan, 46241, Republic of Korea}
\email{sand621@naver.com}
\author[T. H. Kim]{Taehyeong Kim}
\address{Taehyeong Kim\\
	Finance.Fishery.Manufacture Industrial Mathematics Center on Big Data, Pusan National University, Busan, 46241, Republic of Korea}
\email{xogud7936@pusan.ac.kr}
\author[H.-M. Kim]{Hyun-Min Kim}
\address{Hyun-Min Kim\\
	Finance.Fishery.Manufacture Industrial Mathematics Center on Big Data, Pusan National University, Busan, 46241, Republic of Korea}
\email{hyunmin@pusan.ac.kr}
\author[K. M. Kong]{Ki Man Kong}
\address{Ki Man Kong\\
	World Komax Co., Ltd., 1505, Centum Jungang-Ro 48,Haeundae-Gu, Busan, 48059, Republic of Korea}
\email{kennethkong@worldkomax.net}
\author[J. R. Park]{Jeong Rye Park}
\address{Jeong Rye Park\\
	Finance.Fishery.Manufacture Industrial Mathematics Center on Big Data, Pusan National University, Busan, 46241, Republic of Korea}
\email{parkjr@pusan.ac.kr}
\author[J.-H. Seo]{Jong-hyeon Seo}
\address{Jong-hyeon Seo\\
	Chubu University Academy of Emerging Science, Kasugai, 487-0027, Japan}
\email{hyeonni94@gmail.com}
\author[S.-H. Seo]{Sang-hyup Seo$^{\dagger}$}
\address{Sang-hyup Seo\\
	Finance.Fishery.Manufacture Industrial Mathematics Center on Big Data, Pusan National University, Busan, 46241, Republic of Korea}
\email{saibie1677@gmail.com}
\author[S. W. Yoon]{Shin won Yoon}
\address{Shin won Yoon\\
	Finance.Fishery.Manufacture Industrial Mathematics Center on Big Data, Pusan National University, Busan, 46241, Republic of Korea}
\email{ysw0123@pusan.ac.kr}
\thanks{$^{\dagger}$Corresponding author}
%\markleft{G. S. Jang, T. H. Kim, H.-M. Kim, K. M. Kong, J. R. Park, J.-H. Seo, S.-H. Seo, S. W. Yoon}
%\date{Received: date / Accepted: date}
% The correct dates will be entered by the editor

\begin{document}

\begin{abstract}
One of the most classic problems in the manufacturing industry is inventory processing. 
One way to effectively reduce inventory losses is by changing the arrangement of pieces on the printing plates used for offset printing. 
Here we adopt an upper limit of acceptable loss for each plate, before conducting complete enumeration. 
This method dramatically reduces the operating time of the algorithm. 
{\color{blue} Original:The advantage of this method is that there is only change the arrangement of the pieces on the plates.

Suggested: The advantage of this method is that it focuses on changing the arrangment of pieces on one plate at a time. }
\end{abstract}
\subjclass[2010]{68U99, 90C90}
\keywords{Offset Printings \and Combinations with Repetition \and Label Arrangements \and Inventory Managements}

\maketitle

\markleft{G. S. Jang, T. H. Kim, H.-M. Kim, K. M. Kong, J. R. Park, J.-H. Seo, S.-H. Seo, and S. W. Yoon}

\section{Introduction}\label{sec:intro}
A combination with repetition is the number of cases, where $k$ elements are selected from among different $n$ elements allowing repetition \cite{Brualdi2004}. 
It is indicated with the symbol $_{n}H_{k}$ and the following is established.
\begin{equation}
	_{n}H_{k} = _{n+k-1}C_{k} = \frac{(n+k-1)!}{(n-1)!k!}
\end{equation}
For instance, the combination with repetition $_{2}H_{4}$ to select four elements from among two elements A and B comprises the following five cases.
\begin{enumerate}[(1)]
	\item $\textrm{[A, A, A, A]}$ : a list consisting of four A's
	\item $\textrm{[A, A, A, B]}$ : a list consisting of three A's and one B
	\item $\textrm{[A, A, B, B]}$ : a list consisting of two each of A's and B's
	\item $\textrm{[A, B, B, B]}$ : a list consisting of one A and three B's
	\item $\textrm{[B, B, B, B]}$ : a list consisting of four B's
\end{enumerate}
Among the above cases, if we want to obtain three A's and nine B's, then we can choose $[\textrm{A, B, B, B}] \times 3$. 
Also, we consider the following as another case.
\begin{equation}\label{eq:3lists}
	\textrm{[A, A, B, B]} \times 1 + \textrm{[A, B, B, B]} \times 1 + \textrm{[B, B, B, B]} \times 1
\end{equation}
In this case, we can get the three A's and nine B's. 
However, the former case seems to be `better' because there are three different lists in (\ref{eq:3lists}). 
Let us examine another case.
\begin{equation}
\textrm{[A, A, A, A]} \times 1 + \textrm{[B, B, B, B]} \times 3 - \textrm{[A]} \times 1 - \textrm{[B]} \times 3
\end{equation}
$\textrm{[A, B, B, B]} \times 3$ still seems to be `better' because there is no loss. Under the following conditions, $\textrm{[A, B, B, B]} \times 3$ is the `best' method.
\begin{enumerate}[(1)]
	\item Minimize the number of lists.
	\item Minimize the loss of lists.
\end{enumerate}
Offset printing, also called offset lithography, or litho-offset, in commercial printing, 
is a widely used printing technique in which the inked image on a printing plate is printed on a rubber cylinder and then transferred (i.e., offset) to paper or other material. 
The rubber cylinder gives great flexibility, permitting printing on wood, cloth, metal, leather, and rough paper (see Figure \ref{fig:OffsetPrint}) \cite{OffsetPrint}.

\begin{figure}[h!]
	\centering
	\includegraphics[width=7cm]{OffsetPrint.pdf}
	\caption{Offset Printing}
	\label{fig:OffsetPrint}
\end{figure}

Offset printing is one of the most common ways of creating printed materials. A few of its common applications include: newspapers, magazines, brochures, stationery, and books. Compared to other printing methods, offset printing is best suited for economically producing large volumes of high quality prints in a manner that requires little maintenance \cite{Kipphan2001}. 
Several studies focussed on improving the printing process \cite{AlChan, Muscle, Carmo}.
Another important issue for improvement is how to make the initial plates. The above example explores the best plate arrangement for the printing method. 

In the past, production was based on ordering of products from companies and predictions of consumption. 
However, as the internet market has became popular, production systems have been changed by consumers. 
Now, many factories produce only products ordered by consumers.

World Komax is a company that produces labels using offset printing. The labels refer to stickers containing bar-codes attached to garments or shoes as follows (see Figure \ref{fig:AirHuarache}). Each bar-code in the label contains fixed information such as product names and colors, and variable information such as the date of manufacture.

\begin{figure}[h!]
	\centering
	\includegraphics[width=6.5cm]{AirHuarache.pdf}
	\caption{Sneaker Review : Nike Air Huarache}
	\label{fig:AirHuarache}       % Give a unique label
\end{figure}

\noindent
Since the production process of World Komax is not ideal for small quantity batch productions, label losses have increased compared to the past.

Section 2 of this paper will describe the process of label printing using offsets. The modeling of the problem will be carried out in Section 3, and examples to aid problem understanding will be presented in Section 4. The final results of the algorithm will be described in Section 5. 


\section{Offset label printing process}\label{sec:Offset}

\subsection{Label printing process}\label{subsec:LabelPrinting}
Prior to describing the label printing process, we define the following terms first.
\begin{enumerate}[*]
	\item {\bf Plate} : A printing plate for the offset printing (see Figure \ref{fig:PlateLabel})
	\item {\bf Loss} : The number of labels printed in excess of the order-quantity
\end{enumerate}

The offset label printing process is as follows. First, orders are received from customers. Each order includes many types of labels and order-quantities by type (see Figure \ref{fig:Order_Sorting}). Thereafter, offset printing plates are made. Many types of labels are placed on each plate so that many labels are printed at one printing. After the plates are made, more labels than order-quantities are typically produced using each plate. The sheets are then cut to the sizes of labels. In the final process, the labels are collected by type.

\begin{wrapfigure}{r}{4cm}
	\includegraphics[width=4cm]{PlateLabel.pdf}
	\caption{Plate and Label}
	\label{fig:PlateLabel}       % Give a unique label
\end{wrapfigure}

\subsection{Major points for cost saving}\label{subsec:CostSave}
The constraint conditions and major points that will be considered in this paper for cost saving are as follows. First, each type of label should be placed on only one plate. This is to prevent different types of labels from being mixed when collected by type after the printed sheets are cut. Meanwhile, the total number of labels placed on each plate is also constant because the sizes of individual plates are constant and the sizes of labels in one order are also constant. In addition, the number of plates should be minimized because plates are made using molds and the associated costs are high. Finally, the Loss should be minimized because overprinted labels cannot be used and should be entirely discarded. {\color{blue} Note ``Lable" is misspelt in Figure 2}

\subsection{Sorting reports output program}\label{subsec:SortProgram}
Plate fabrication and the use of printing paper incur costs. To reduce the costs, appropriate methods are used to output sorting reports that describe how the labels of each order should be arranged on plates. The plate makers produce plates according to the instructions in the sorting reports (see Figure \ref{fig:Order_Sorting}).

Since the existing sorting report output method is not suitable for small quantity batch production systems, an improved algorithm is needed. Therefore, this study was conducted to develop new algorithms suitable for small quantity batch production systems.

\begin{figure}[h!]
	\centering
	\resizebox{14cm}{7cm}{\includegraphics{OrderForm.pdf}\includegraphics{SortingReport.pdf}}\\
	\caption{An Order Form(left) and a Sorting Report(right) \\(Source: World Komax Co., Ltd.)}
	\label{fig:Order_Sorting}       % Give a unique label
\end{figure}

\section{Modeling and flowchart}\label{sec:Modeling}
To formulate our problem as a mathematical optimization problem, we first need to define our notation.
\begin{enumerate}[$\bullet$]
	\item Let $I$ be a set of products.
	\item For each product $i \in I$, let $b_{i}$ be the quantity of product specified in the order.
	\item Let ${\bf b}=(b_{i}|i \in I)$ be a vector of order-quantities.
	\item $k$ is the total number of labels that can be placed in one Plate.
	\item $\pi$ is a partition of $I$ such that $1 \leq  |P| \leq k$ for any $P \in \pi$. 
	\item Let $\Gamma_{\pi}$ be a set of matrices $A \in {\rm Mat}_{\pi \times I}(\mathbf{Z})$ that satisfy the following:
	\begin{itemize}
		\item[-] For all $(P,i) \in \pi \times I$, $A_{P,i} \geq 0$, and $A_{P,i} = 0$ if and only if $i \notin P$.
		\item[-] For each $P \in \pi$, $\sum_{i \in P}A_{P, i} = k$
\end{itemize}
\end{enumerate}

Using the above notation,
\begin{equation}\label{eq:NumPlate}
	\left\lceil \max \left\{ \left.\frac{b_{i}}{A_{P,i}}\right|i \in P \right\} \right\rceil
\end{equation}
is the printing number of Plate $P$, where $\lceil~\rceil$ means the ceiling. 
Assume that $\alpha$ is the cost to produce one Plate, and $\beta$ is the cost of the loss of one label. 
Then, our goal is to obtain the following 
\begin{equation}\label{eq:TotalCost}
	\min_{\pi} \{ \alpha|\pi| + \beta E_{A,b} | A \in \Gamma_{\pi} \}
\end{equation}
where
\begin{equation}\label{eq:TotalLoss}
	E_{A,b} = \sum_{P \in \pi} \sum_{i \in I} \left( \left\lceil \max \left\{ \left.\frac{b_{i}}{A_{P,i}}\right|i \in P \right\} \right\rceil \cdot A_{P,i} - b_{i} \right)
\end{equation}
means the total number of labels lost.

Naturally, the complete enumeration using combinations with repetition is the surest way. 
However, the problem with this method is that it takes too much time.
For instance, when $n=65$ and $k = 24$, the combination with repetition $_{65}H_{24}$ comprises about $2.36 \times 10^{21}$ cases. The calculation of these cases takes more than 658 hours, that is, more than 27 days using a super computer that can calculate $10^{15}$ partitions per second. This is a very long computation time given that there are limits on the time from the date of receipt of orders to the delivery date,

In this algorithm, this problem was solved by introducing positive integers as thresholds. 
Thresholds correspond to the allowed amount of label losses occurring in each plate. 
Adopting partitions that do not exceed the thresholds dramatically reduces the time taken.
Based on the foregoing, a flowchart of the algorithm can be set forth as follows (see Figure \ref{fig:MFChart}).

\begin{figure}[h!]
	\centering
	\includegraphics[width=4cm]{MainFChart.pdf}
	\caption{The Main Flowchart}
	\label{fig:MFChart}       % Give a unique label
\end{figure}

In Figure \ref{fig:MFChart}, $z$ is the threshold and several other symbols are explained above.
This algorithm outputs a matrix $A=\bigoplus\limits_{P} A_{P,i}$ containing the label of each product when $I$ and ${\bf b}$ have been inputted for $z$, $k$, $num$. 
One of the results of ${loop}(k, z, num)$, are the products $P$ contained on a Plate (see Figure \ref{fig:SFChart}). 
By removing $P$ from $I$, we ensure that different plates do not contain the same product.
The algorithm repeats until $I$ is empty.

\begin{figure}[h!]
	\centering
	\includegraphics[width=8cm]{SubFChart.pdf}
	\caption{The Flowchart of ${loop}(k, z, num)$}
	\label{fig:SFChart}       % Give a unique label
\end{figure}

Meanwhile, when the ${loop}(k, z, num)$ function is encountered, the flowchart in Figure \ref{fig:SFChart} should be followed.
First, the ${Part}(k, num)$ function finds partitions using combination with repetition $_{num}H_{k}$ 
%(method to select $k$ pieces of products from $num$ pieces of products allowing repetition)
that are indicated in the form of the list {\it Part\underline{ }list}.
For example, let $k=6$ and $num=3$, then $\text{\it Part\underline{ }list}={Part}(6, 3) = \{ [4,1,1],[3,2,1],[2,2,2] \}$.
$N$ is a printing number calculated according to \eqref{eq:NumPlate}.
An appropriate $P$ that has $num$ different products is selected from $I$ and the loss is obtained using {\it Part\underline{ }list} and the printing number.

If the loss exceeds the threshold $z$, another {\it Part\underline{ }list} will be selected and the foregoing will be repeated  while adjusting $num$ until the threshold $z$ is not exceeded.
As a result of this process, $A_{P,i}$ whose loss does not exceed the threshold is obtained.% (see Figure \ref{fig:SFChart}).



\section{Examples}\label{sec:Exam}
These examples are described to aid with understanding of the problem. For the next two examples, we assume that $k$ is equal to 4.

\begin{example}
	Assume that $I=\{1,2,3\}$, and the order-quantity vector ${\bf b}=(50,30,20)$.
	
	We consider a partition $\pi = \{\{1,2\}, \{3\}\}$ for the order-quantity vector ${\bf b}$. 
	Without loss of generality, assume that $P_{1} = \{1,2\}, P_{2} = \{3\}$. Since $k = 4$, the matrix $A$ can be found as follows.
	
	\begin{equation}
		A = \left(\begin{array}{ccc}2 & 2 & 0 \\ 0 & 0 & 4 \end{array}\right)
	\end{equation}
	
	In this case, the printing numbers (\ref{eq:NumPlate}) of $P_1$ and $P_2$ are as follows, respectively.
	\begin{equation}
		\left\lceil \max\left\{ \left. \frac{b_{i}}{A_{P_{1},i}} \right| i \in P_{1} \right\} \right\rceil = \left\lceil \max \left\{ \frac{50}{2}, \frac{30}{2} \right\} \right\rceil = 25
	\end{equation}
	and
	\begin{equation}
	\left\lceil \max\left\{ \left. \frac{b_{i}}{A_{P_{2},i}} \right| i \in P_{2} \right\} \right\rceil = \left\lceil \max \left\{ \frac{20}{4} \right\} \right\rceil = 5.
	\end{equation}
	The situation is illustrated in Figure \ref{fig:ex11}.
	
	\begin{figure}[h!]
		\centering
		\includegraphics[width=6cm]{ex11.pdf}
		\caption{}
		\label{fig:ex11}       % Give a unique label
	\end{figure}
	
	It can be seen that the total Loss $E_{A,{\bf b}}$ is 20.
	
	Now we consider a new partition $\pi = \{\{1,2,3\}\}$. In this case, $A = (\begin{array}{ccc}2 & 1 & 1 \end{array})$, and printing number (\ref{eq:NumPlate}) is 
	\begin{equation}
	\left\lceil \max\left\{ \left. \frac{b_{i}}{A_{P,i}} \right| i \in P \right\} \right\rceil = \left\lceil \max \left\{ \frac{50}{2}, \frac{30}{1}, \frac{20}{1} \right\} \right\rceil = 30.
	\end{equation}
	In addition, it can be easily seen that the total Loss $E_{A,{\bf b}}$ is 20 (see Figure \ref{fig:ex12}).
	The array shown in Figure \ref{fig:ex12} is a more efficient because its Loss is the same but its number of plates is smaller.
	\begin{figure}[h!]
		\centering
		\includegraphics[width=4cm]{ex12.pdf}
		\caption{}
		\label{fig:ex12}       % Give a unique label
	\end{figure}
\end{example}

\begin{example}
	Assume that $I=\{1,2\}$, and the order-quantity vector ${\bf b}=(50,20)$.
	
	We consider a partition $\pi = \{\{1,2\}\}$ for the quantity vector ${\bf b}$. Since $k = 4$, 
	the matrix $A = (\begin{array}{cc}2 & 2\end{array})$, and the printing number (\ref{eq:NumPlate}) is 
	\begin{equation}
	\left\lceil \max\left\{ \left. \frac{b_{i}}{A_{P,i}} \right| i \in P \right\} \right\rceil = \left\lceil \max \left\{ \frac{50}{2}, \frac{20}{2} \right\} \right\rceil = 25.
	\end{equation}
	The total Loss $E_{A,{\bf b}}$ is 30 (see Figure \ref{fig:ex21}).
	
	\begin{figure}[h!]
		\centering
		\includegraphics[width=3.5cm]{ex21.pdf}
		\caption{}
		\label{fig:ex21}       % Give a unique label
	\end{figure}
	
	For the same partition $\pi$, the matrix $A = (\begin{array}{cc}3 & 1\end{array})$ can be considered. In this case, the printing number (\ref{eq:NumPlate}) is 
	\begin{equation}
	\left\lceil \max\left\{ \left. \frac{b_{i}}{A_{P,i}} \right| i \in P \right\} \right\rceil = \left\lceil \max \left\{ \frac{50}{3}, \frac{20}{1} \right\} \right\rceil = 20.
	\end{equation}
	and the total Loss $E_{A,{\bf b}}$ is 10 (see Figure \ref{fig:ex22}).
	The array shown in Figure \ref{fig:ex22} is more efficient because the number of plates are the same but fewer losses occur.
	\begin{figure}[h!]
		\centering
		\includegraphics[width=3.5cm]{ex22.pdf}
		\caption{}
		\label{fig:ex22}       % Give a unique label
	\end{figure}
\end{example}


If there are many products with large order-quantities, a lot of iteration is needed. Now we consider a real problem from World Komax. In this case, $k=18$, $\alpha=300$, $\beta=1$. Table ~\ref{Tab:1} is an actual sorting report of an order from World Komax.

\begin{table}[h!]
	\centering
	\caption{The sorting report of World Komax}
	\begin{tabular}{ c|c|c|c|c|c|c } 
		\hline
		Plate&product&order-quantity&psc&printing-number&production&Loss\\
		\hline
		\multirow{5}*{1}
		&$1$&59&2&\multirow{5}*{32}&64&5\\
		\cline{2-4}
		\cline{6-7}
		&$2$&156&5&&160&4\\
		\cline{2-4}
		\cline{6-7}
		&$3$&9&1&&32&23\\
		\cline{2-4}
		\cline{6-7}
		&$4$&162&6&&192&30\\
		\cline{2-4}
		\cline{6-7}
		&$5$&102&4&&128&26\\
		
		\hline
		\multirow{5}*{2}
		&$6$&1059&5&\multirow{5}*{228}&1140&81\\
		\cline{2-4}
		\cline{6-7}
		&$7$&886&4&&912&26\\
		\cline{2-4}
		\cline{6-7}
		&$8$&228&1&&228&0\\
		\cline{2-4}
		\cline{6-7}
		&$9$&832&4&&912&80\\
		\cline{2-4}
		\cline{6-7}
		&$10$&862&4&&912&50\\
		
		\hline
		\multirow{4}*{3}
		&$11$&532&4&\multirow{4}*{133}&532&0\\
		\cline{2-4}
		\cline{6-7}
		&$12$&532&5&&665&133\\
		\cline{2-4}
		\cline{6-7}
		&$13$&482&4&&532&50\\
		\cline{2-4}
		\cline{6-7}
		&$14$&582&5&&665&103\\
		\hline
	\end{tabular}
	\label{Tab:1}	
\end{table}

As we can see from Table \ref{Tab:1}, there are $14$ products and the largest order-quantity is $1059$. It uses $3$ plates and has a total Loss of $611$. The total cost is $1511$.
\begin{example}
We obtain another partition of the sorting report using our algorithms. Table \ref{Tab:2} shows this result.
%Table \ref{Tab:2} shows the result calculated by our algorithm.

\begin{table}[h!]
	\centering
	\caption{The result of our algorithm}
	\begin{tabular}{ c|c|c|c|c|c|c } 
		\hline
		Plate&product&order-quantity&psc&printing-number&production&Loss\\
		\hline
		\multirow{6}*{1}
		&$7$&886&5&\multirow{6}*{178}&890&4\\
		\cline{2-4}
		\cline{6-7}
		&$10$&862&5&&890&28\\
		\cline{2-4}
		\cline{6-7}
		&$11$&532&3&&534&2\\
		\cline{2-4}
		\cline{6-7}
		&$12$&532&3&&534&2\\
		\cline{2-4}
		\cline{6-7}
		&$4$&162&1&&178&16\\
		\cline{2-4}
		\cline{6-7}
		&$2$&156&1&&178&22\\
		
		\hline
		\multirow{5}*{2}
		&$6$&1059&9&\multirow{5}*{118}&1062&3\\
		\cline{2-4}
		\cline{6-7}
		&$14$&562&5&&590&28\\
		\cline{2-4}
		\cline{6-7}
		&$8$&228&2&&236&8\\
		\cline{2-4}
		\cline{6-7}
		&$5$&102&1&&118&16\\
		\cline{2-4}
		\cline{6-7}
		&$1$&59&1&&118&59\\
		
		
		\hline
		\multirow{2}*{3}
		&$9$&832&11&\multirow{2}*{76}&836&4\\
		\cline{2-4}
		\cline{6-7}
		&$13$&482&7&&532&50\\
		
		\hline
		4&$3$&9&18&1&18&9\\
		\hline
	\end{tabular}
\label{Tab:2}	
  \end{table}

\noindent	
The matrix $A$ can be found as follows.
\begin{equation*}
\begin{array}{rc}
&\text{column index}\\
A\,\,=&
\begin{blockarray}{cccccccccccccc}
7 & 10 & 11 & 12 & 4 & 2 & 6 & 14 & 8 & 5 & 1 & 9 & 13 & 3\\
&&&&&&&&&&&&&\\
\begin{block}{(cccccccccccccc)}
5 & 5 & 3 & 3 & 1 & 1 & 0 & 0 & 0 & 0 & 0 & 0 & 0 & 0 \\
0 & 0 & 0 & 0 & 0 & 0 & 9 & 5 & 2 & 1 & 1 & 0 & 0 & 0 \\
0 & 0 & 0 & 0 & 0 & 0 & 0 & 0 & 0 & 0 & 0 & 11& 7 & 0 \\
0 & 0 & 0 & 0 & 0 & 0 & 0 & 0 & 0 & 0 & 0 & 0 & 0 & 18\\
\end{block}
\end{blockarray}
\end{array}
\end{equation*}
\noindent
The total Loss $E_{A,b}$ is $251$. Hence the total cost is $\alpha|\pi|+\beta E_{A,b}=300 \times 5+1 \times 251 =1451$. We use one more Plate, but the total cost is reduced by $60$.
%	As a result, we used the same number of plates, but because of less loss, our algorithm has less total cost.
	

	
\end{example}



\section{Result}\label{sec:Result}

Each sorting report includes the number of losses corresponding to one plate, so the total cost can be calculated.
We used 82 sorting report samples. {\color{blue} Original: ``The algorithm can be improved for every sample data." I might have suggested replacing this with "The algorithm reduces the cost for every sorting report sample." but this is incorrect, as demonstrated by sample 15. Perhaps it's better to delete this sentence.}
The total cost was reduced from a minimum of -6.85\%(sample no. 15) to a maximum of 27.5\%(sample no. 74), see Figure \ref{fig:Comparing}.

\begin{figure}[h!]
	\centering
%	\includegraphics[width=\linewidth]{Comparing.pdf}
	\includegraphics[width=11cm]{Graph_2.pdf}
	\caption{Comparing the Results}
	\label{fig:Comparing}       % Give a unique label
\end{figure}

We used the {\it paired t-test}\cite{Rice} to verify the efficiency of the algorithm. The paired t-test is one of the two sample t-tests that verifies whether two groups are different. The two populations are as follows.

\begin{enumerate}[$\bullet$]
	\item population1: total cost before applying the algorithm
	\item population2: total cost after applying the algorithm
	\item sample1: sample of 82 items from population1
	\item sample2: sample of 82 items from population2
\end{enumerate}

In order to proceed with the two-sample t-test, the two groups have to satisfy normality and homoscedasticity. The $82$ sample data from the two populations, satisfy normality by the {\it central limit theorem} \cite{Durrett}.
In addition, we verified the homoscedasticity of the two samples using {\it var.test} of R. 
R is a programming language and free software environment for statistical computing and graphics. % supported by the R Foundation for Statistical Computing. 
%The R language is widely used among statisticians and data miners for developing statistical software and data analysis.



The null hypothesis $(H_{0})$ of var.test is `the variances of the two groups are equal', and the alternative hypothesis $(H_{1})$ is `the variances of the two groups are different'. 
If the $p$-value is below the significance level, the null hypothesis $(H_{0})$ is rejected and if the $p$-value is not lower than the significance level, the alternative hypothesis $(H_{1})$ is rejected.  The result of var.test with a significance level of 0.05 is as follows (see Figure \ref{fig:vartest}).

\begin{figure}[h!]
	\centering
	\fbox{\includegraphics[width=8cm]{vartest.png}}
	\caption{var.test of R}
	\label{fig:vartest}       % Give a unique label
\end{figure}

Since the $p$-value is not lower than the significance level(0.05), the alternative hypothesis $(H_{1})$ is rejected and the null hypothesis $(H_{0})$ is accepted. We conclude that the variances of the two groups can be equal.

Since the two groups satisfy normality and homoscedasticity, we verified whether the difference between the two groups is significant through a paired t-test. In this test, the null hypothesis $(H_{0})$ is `the total cost will be the same after applying the algorithm', and the alternative hypothesis $(H_{1})$ is `the total cost will be reduced after applying the algorithm.' The result of the paired t-test with a significance level of 0.05 is as follows (see Figure \ref{fig:ttest}).

\begin{figure}[h!]
	\centering
	\fbox{\includegraphics[width=8cm]{ttest.png}}
	\caption{t.test of R}
	\label{fig:ttest}       % Give a unique label
\end{figure}

Since the $p$-value is below the significance level(0.05), the null hypothesis $(H_{0})$ is rejected and the alternative hypothesis $(H_{1})$ is accepted. We conclude that the difference between the two groups is significant.

Finally, we check how the efficiency varies with the number of products (the size of $I$), cf. Section \ref{sec:Modeling}.
We see that the efficiency increases as the number of products increases in Figure \ref{fig:LinearFitting}.
For each of the 82 samples, we calculated the efficiency according to the following formula.
\begin{equation}
	\textrm{Efficiency} = \left[1-\frac{\textrm{Improvement~Total~Cost}}{\textrm{Previous~Total~Cost}}\right]\times 100
\end{equation}
The result of the linear fit shows that the efficiency and the number of products have a positive correlation.
Thus, we see that the improved algorithm has better results for a larger number of products.
This suggests that the algorithm is well suited to small quantity batch production.


\begin{figure}[h!]
	\centering
	%	\includegraphics[width=\linewidth]{Comparing.pdf}
	\includegraphics[width=7.5cm]{Graph_3.png}
	\caption{The linear fitting for the efficiency compared with the number of products}
	\label{fig:LinearFitting}       % Give a unique label
\end{figure}


{\bf Notice.} Please note that the detailed ideas of the algorithm cannot be described due to confidentiality agreements with World Komax.

~\\
\indent{{\bf Acknowledgements.} This work was supported by the National Research Foundation of Korea (NRF)
grant funded by the Korean Government (MSIP) (NRF-2017R1A5A1015722, NRF-2018R1D1A1B07048197). 
%The authors thank the anonymous referees for providing very useful suggestions for improving this paper.
The authors extend thanks to Professor Hyun-Min Kim and Professor Sang-il Kim of Pusan National University who provided much support and advice for the writing of this paper.}

\bibliographystyle{plain}
\bibliography{mybibfile}

\end{document}
% end of file template.tex

g